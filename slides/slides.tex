% ----------------------------------------------------------------------------------------------------------------------------------
% High Performance pgBackRest
%
% Build from the Vagrant VM:
% cd /talk/slides && make -f /template/Makefile
% ----------------------------------------------------------------------------------------------------------------------------------
\def\mytitle{pgBackRest Overview}
\def\mysubject{}
\def\myevent{Crunchy Storm}
\def\myauthor{David Steele}
\def\myemail{}
\def\mydate{November 28, 2018}

% Suppres navigation bars
\def\mysuppressnav{}

% Include Crunchy template
\def\mytemplatepath{/template/}
\input{\mytemplatepath crunchy-template.tex}

% Agenda
\begin{frame}
    \frametitle{Agenda}
    \tableofcontents
\end{frame}

\section{Introduction}

\begin{frame}
    \frametitle{About David}

    \begin{itemize}
        \item Principal Architect at Crunchy Data, the Trusted Open Source Enterprise PostgreSQL Leader.
        \item Actively developing with PostgreSQL since 1999.
        \item PostgreSQL Contributor.
        \item Primary author of pgBackRest and co-author of pgAudit.
    \end{itemize}
\end{frame}

\begin{frame}
    \frametitle{What is pgBackRest?}

    pgBackRest aims to be a simple, reliable backup and restore system that can seamlessly scale up to the largest databases and workloads.\pause\vspace{1em}

    pgBackRest has a strong emphasis on performance, including:

    \begin{itemize}
        \item Parallel/asynchronous operation for all core commands\pause
        \item Backup from Standby\pause
        \item Advanced and flexible configuration for tuning specific commands
    \end{itemize}
\end{frame}

\begin{frame}
    \frametitle{Coming Up}

    pgBackRest is currently being rewritten in pure C.\pause\vspace{1em}

    pgBackRest has a strong emphasis on performance, including:

    \begin{itemize}
        \item Parallel/asynchronous operation for all core commands\pause
        \item Backup from Standby\pause
        \item Advanced and flexible configuration for tuning specific commands
    \end{itemize}
\end{frame}

\begin{frame}
    \frametitle{Questions?}

    website: \url{http://www.pgbackrest.org}\\
    \vspace{1em}
    email: \href{mailto:david@pgbackrest.org}{david@pgbackrest.org} \\
    email: \href{mailto:david@crunchydata.com}{david@crunchydata.com}\\
    \vspace{1em}
    releases: \url{https://github.com/pgbackrest/pgbackrest/releases}\\
    \vspace{1em}
    slides \& demo: \url{https://github.com/dwsteele/conference/releases}\\
\end{frame}

% End document
\end{document}
