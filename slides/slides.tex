% ----------------------------------------------------------------------------------------------------------------------------------
% High Performance pgBackRest
%
% Build from the Vagrant VM:
% cd /talk/slides && make -f /template/Makefile
% ----------------------------------------------------------------------------------------------------------------------------------
\def\mytitle{pgBackRest Overview}
\def\mysubject{}
\def\myevent{Crunchy Storm}
\def\myauthor{David Steele}
\def\myemail{}
\def\mydate{November 28, 2018}

% Suppres navigation bars
\def\mysuppressnav{}

% Include Crunchy template
\def\mytemplatepath{/template/}
\input{\mytemplatepath crunchy-template.tex}

\begin{frame}
    \frametitle{About David}

    \begin{itemize}
        \item Principal Architect at Crunchy Data, the Trusted Open Source Enterprise PostgreSQL Leader.
        \item Actively developing with PostgreSQL since 1999.
        \item PostgreSQL Contributor.
        \item Primary author of pgBackRest and co-author of pgAudit.
    \end{itemize}
\end{frame}

\begin{frame}
    \frametitle{What is pgBackRest?}

    pgBackRest aims to be a simple, reliable backup and restore system that can seamlessly scale up to the largest databases and workloads.\pause
\end{frame}

\begin{frame}
    \frametitle{Performance}

    pgBackRest has a strong emphasis on performance:

    \begin{itemize}
        \item Parallel/asynchronous operation for all core commands\pause
        \item Backup from Standby\pause
        \item Advanced and flexible configuration for tuning specific commands
        \item Delta restore mode using checksums\pause
    \end{itemize}
\end{frame}

\begin{frame}
    \frametitle{Reliability}

    pgBackRest strives to avoid corruption and misconfiguration:

    \begin{itemize}
        \item All files/manifests are checksummed
        \item Checksums are tested on every operation
        \item Version and system ID are used to avoid crossing repositories
        \item Archive command that de-dupes and does not overwrite existing WAL
    \end{itemize}
\end{frame}

\begin{frame}
    \frametitle{Code Coverage}

    pgBackRest has unparalleled code coverage:

    \begin{itemize}
        \item 99.6\% line/branch/condition coverage in the C code (unit tests)
        \item 93.1\% line/branch/condition coverage in the Perl code (unit/integration tests)
        \item Each uncovered line/branch/condition is documented with the reason it is not covered
        \item New C patches must have 100\% coverage (including documented exceptions) to be accepted
    \end{itemize}
\end{frame}

\begin{frame}
    \frametitle{The Future}

    pgBackRest is currently being migrated to C and code coverage is being increased to 100\% (including documented exceptions).\vspace{1em}

    Some future features under consideration:

    \begin{itemize}
        \item Protocol layer over SSL (no SSH required)
        \item Page-level incremental
        \item Windows support
    \end{itemize}
\end{frame}

\begin{frame}
    \frametitle{Questions?}

    website: \url{http://www.pgbackrest.org}\\
    \vspace{1em}
    email: \href{mailto:david@crunchydata.com}{david@crunchydata.com}\\
    \vspace{1em}
    source: \url{https://github.com/pgbackrest/pgbackrest}
\end{frame}

% End document
\end{document}
